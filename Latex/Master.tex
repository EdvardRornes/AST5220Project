% ****** Start of file apssamp.tex ******
%
%   This file is part of the APS files in the REVTeX 4.2 distribution.
%   Version 4.2a of REVTeX, December 2014
%
%   Copyright (c) 2014 The American Physical Society.
%
%   See the REVTeX 4 README file for restrictions and more information.
%
% TeX'ing this file requires that you have AMS-LaTeX 2.0 installed
% as well as the rest of the prerequisites for REVTeX 4.2
%
% See the REVTeX 4 README file
% It also requires running BibTeX. The commands are as follows:
%
%  1)  latex apssamp.tex
%  2)  bibtex apssamp
%  3)  latex apssamp.tex
%  4)  latex apssamp.tex
%
\documentclass[%
reprint,
%superscriptaddress,
%groupedaddress,
%unsortedaddress,
%runinaddress,
%frontmatterverbose, 
%preprint,
%preprintnumbers,
%nofootinbib,
%nobibnotes,
%bibnotes,
 amsmath,amssymb,
 aps,
%pra,
%prb,
%rmp,
%prstab,
%prstper,
%floatfix,
]{revtex4-2}

\usepackage{subfiles}
\usepackage{graphicx}% Include figure files
\usepackage{dcolumn}% Align table columns on decimal point
\usepackage{bm}% bold math
\usepackage{float}
\usepackage{mathtools}
\usepackage{xcolor}
%\usepackage{hyperref}% add hypertext capabilities
%\usepackage[mathlines]{lineno}% Enable numbering of text and display math
%\linenumbers\relax % Commence numbering lines

%\usepackage[showframe,%Uncomment any one of the following lines to test 
%%scale=0.7, marginratio={1:1, 2:3}, ignoreall,% default settings
%%text={7in,10in},centering,
%%margin=1.5in,
%%total={6.5in,8.75in}, top=1.2in, left=0.9in, includefoot,
%%height=10in,a5paper,hmargin={3cm,0.8in},
%]{geometry}

\newcommand{\Hp}{\mathcal{H}}
\renewcommand{\thesection}{\arabic{section}}
\renewcommand{\thesubsection}{\thesection.\arabic{subsection}}
\renewcommand{\thesubsubsection}{\thesubsection.\arabic{subsubsection}}

\begin{document}

\title{The Large Scale Structure of the Cosmic Microwave Background}

\author{E. B. Rørnes}
 \email{e.b.rornes@fys.uio.no}
\affiliation{Institute of Physics, University of Oslo,\\0371 Oslo,  Norway}

\date{\today}

\begin{abstract}
In this paper we consider the large scale structure of the universe and the Cosmic Microwave Background (CMB) fluctuations...
\end{abstract}

\keywords{cosmic microwave background  --	large-scale structure of Universe}
\maketitle

%\tableofcontents

\section*{Introduction}

This project was built on the template provided from (cite) \color{red}[1. Should I cite the github which includes the template? If so, is there any special form for this?] \color{black} which already contains a large part of the structure behind the numerical parts of the project. In the entire paper the subscript 0 denotes today's values... (not a final introduction)

\color{red}Note that I will fix the font size on the figures to match (or at least be similar) to the text size in the main report at some later time. All figures are vector drawn (with the exception of the scatterplot which is rasterized due to large amounts of data) so if it is too small to read you can zoom in if needed.\color{black}




% Include each milestone file

% Milestone I
\subfile{Milestone1/milestone1}

% Milestone II
\subfile{Milestone2/milestone2}

% Milestone III
\subfile{Milestone3/milestone3}

% Milestone IV
\subfile{Milestone4/milestone4}





\begin{acknowledgements}
	...
\end{acknowledgements}

% Bibliography

\bibliographystyle{JHEP}
\bibliography{Master}

\end{document}
